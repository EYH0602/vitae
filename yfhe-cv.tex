\documentclass[margin,line]{res}


\oddsidemargin -.5in
\evensidemargin -.5in
\textwidth=6.0in
\itemsep=0in
\parsep=0in
% if using pdflatex:
%\setlength{\pdfpagewidth}{\paperwidth}
%\setlength{\pdfpageheight}{\paperheight} 

\newenvironment{list1}{
  \begin{list}{\ding{113}}{%
      \setlength{\itemsep}{0in}
      \setlength{\parsep}{0in} \setlength{\parskip}{0in}
      \setlength{\topsep}{0in} \setlength{\partopsep}{0in} 
      \setlength{\leftmargin}{0.17in}}}{\end{list}}
\newenvironment{list2}{
  \begin{list}{$\bullet$}{%
      \setlength{\itemsep}{0in}
      \setlength{\parsep}{0in} \setlength{\parskip}{0in}
      \setlength{\topsep}{0in} \setlength{\partopsep}{0in} 
      \setlength{\leftmargin}{0.2in}}}{\end{list}}

\usepackage{array, booktabs, caption}
\usepackage{makecell}
\newcommand*{\nl}{\newline}

\begin{document}

\name{Yifeng He \vspace*{.1in}}

\begin{resume}
\section{\sc Contact Information}
\vspace{.05in}
\textbf{Voice:} (561) 990-6808 |
\textbf{E-mail:} yfhe@ucdavis.edu |
\textbf{WWW:} https://eyh0602.github.io/
% \begin{tabular}{@{}p{2in}p{4in}}
% Baker Hall 232             & {\it Voice:}  (412) 268-6276 \\            
% Department of Statistics   & {\it Fax:}    (412) 268-7828 \\         
% Carnegie Mellon University & {\it E-mail:}  paciorek@stat.cmu.edu\\       
% Pittsburgh, PA  15213 USA  & {\it WWW:} www.stat.cmu.edu/\verb+~+paciorek \\     
% \end{tabular}


\section{\sc Research Interests}
programming languages for AI, hybrid recommender systems, functional programming,
formal methods for CS, image classification,
collaborative filtering, music-related machine learning and algorithms

\section{\sc Education}
{\bf University of California, Davis}, Davis, California, USA. GPA: 3.887\\
%{\em Department of Mathematics and Statistics} 
\vspace*{-.1in}
\begin{list1}
\item[] B.S., Computer Science, 2019 -- Present (Anticipated Graduation date: 06/2023)
\item[] B.S., Applied Mathematics, 2019 -- Present (Anticipated Graduation date: 06/2023)
\end{list1}


\section{\sc Honors and Awards} 
Dean's Honor List, 2019, 2020, 2021, 2022

%\vspace*{-2.5mm}
%NSF Vertical Integration of Research and Education in Statistics and
%Mathematical Sciences\\ (VIGRE) teaching fellowship.
%
% \vspace*{-2.5mm}
% Carleton College: graduated Magna Cum Laude, Honors in Biology, Phi Beta Kappa, 1993

% \section{\sc Academic Experience}
% {\bf Carnegie Mellon University}, Pittsburgh, Pennsylvania USA

% \vspace{-.3cm}
% {\em Graduate Student} \hfill {\bf August, 1998 - present}\\
% Includes current Ph.D.~research, Ph.D.~and Masters level coursework and
% research/consulting projects.

\section{\sc Publications}
\textbf{He, Y}, (in press), \textit{Big Data and Deep Learning Techniques Applied in Intelligent Recommender Systems}, PPRAI 2022.

\section{\sc Professional Experience}

{\bf ByteDance} \hfill {\bf 04/2021 -- 08/2021} \\
\textit{Software Engineering Intern}, {Income Platform Team}
\begin{itemize}
	\item Used microservice technic to connect all parts of the author income settlement bushiness.
	\item Transformed author-relation data architecture design from relational database (SQL) to graph database (Gremlin) to allow better efficiency for the business model.
	\item Refactored the income calculation control process with visitor design pattern in Python 3 to allow better extendability and maintainability.
	\item Used better modular design to allow easier change of calculation strategy by product manager.
\end{itemize}

\vspace{-.3cm}

{\bf Xigua Video} \hfill {\bf 05/2021 -- 06/2021} \\
\textit{Software Engineering Intern}, {Author Experience Team}
\begin{itemize}
	\item Created a data cleaner script with ORM to maintain the size and readability of online data settlement table (about 5 billion rows) so that all services built on top of that table have reasonable performance.
	\item Created the offline flow of Medium Video Encouragement Project for weekly data calculation, and build the interface for front-end (web and mobile app) to display the data visualization.
\end{itemize}
\vspace{-.3cm}

{\bf HackerHub}, UC Davis Club \hfill {\bf 07/2020 -- Present} \\
\textit{Technical Officer - Co-President - President}
\begin{itemize}
	\item Organized and leads the Code Jam Competition on various topics.
	\item Taught in introductory programming workshops in tops: Assembly, Functional Programming.
\end{itemize}
\vspace{-.3cm}

\section{\sc Projects}
{\bf CourseReco} \hfill {\bf 09/2022 -- 12/2022 } \\
\begin{itemize}
	\item Designed the overall system architecture
	\item Led the programming for API server and recommender engine
	\item Negotiated with SchedGo (third-party) for data service
\end{itemize}
\vspace{-.3cm}

{\bf dcash-server} \hfill {\bf 05/2021 -- 07/2021} \\
\begin{itemize}
	\item Created a multi-threaded API server using C++, allowing user to create account, deposit, and transfer,
	\item Used MySQL to store and maintain user data.
	\item Made API calls to the Stripe API server to handle credit card information.
\end{itemize}
\vspace{-.3cm}


{\bf Music Genre Classifier} \hfill {\bf 05/2022 -- 06/2022} \\
\begin{itemize}
	\item Processed music samples into spectrogram by Short-time Fourier transform
	\item Designed the appropriate model (CNN) to classify spectrograms into category
	\item Analyzed the resulting model and test outputs with saliency maps
\end{itemize}
\vspace{-.3cm}

{\bf ImageOrientation} \hfill {\bf 03/2022 -- 04/2022} \\
\begin{itemize}
	\item Pre-processed image data by rotating them with random generated angles, and assign these angles as label
	\item Designed the appropriate CNN for this regression task, test and improve the model
	\item Applied Hyper-parameter tuning based on train, validation, and test results to achieve better performance
\end{itemize}
\vspace{-.3cm}

{\bf Genshine Impact Gacha Analyzer} \hfill {\bf 08/2021 -- 09/2021} \\
\begin{itemize}
	\item Designed fetching process of gacha data from MiHoYo and process into different categories.
	\item Stored data into local database automatically, write into excel for data analysis by option.
	\item Generated text or graph visualization report from data analyze result
\end{itemize}
\vspace{-.3cm}


{\bf permualgebra} \hfill {\bf Dec. 2020 -- Jan. 2021} \\
\begin{itemize}
  \item Python package that allows calculation of algebraic permutations.
\end{itemize}
\vspace{-.3cm}

{\bf Fishbone++} \\
\begin{itemize}
  \item a delightful oh-my-zsh theme, written in shell script.
\end{itemize}



\section{\sc Computer Skills} 
\begin{list2}
\item Languages: Python $\geq$ C $\approx$ C++ $\geq$ R $\geq$ Haskell $\geq$ js $\approx$ Rust $>$ Java $\approx$ SQL
\end{list2}



\end{resume}
\end{document}




