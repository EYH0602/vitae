\documentclass[margin,line]{res}

\usepackage{hyperref}
\usepackage{bibentry}

\oddsidemargin -.5in
\evensidemargin -.5in
\textwidth=6.0in
\itemsep=0in
\parsep=0in
% if using pdflatex:
%\setlength{\pdfpagewidth}{\paperwidth}
%\setlength{\pdfpageheight}{\paperheight} 

\newenvironment{list1}{
  \begin{list}{\ding{113}}{%
      \setlength{\itemsep}{0in}
      \setlength{\parsep}{0in} \setlength{\parskip}{0in}
      \setlength{\topsep}{0in} \setlength{\partopsep}{0in} 
      \setlength{\leftmargin}{0.17in}}}{\end{list}}
\newenvironment{list2}{
  \begin{list}{$\bullet$}{%
      \setlength{\itemsep}{0in}
      \setlength{\parsep}{0in} \setlength{\parskip}{0in}
      \setlength{\topsep}{0in} \setlength{\partopsep}{0in} 
      \setlength{\leftmargin}{0.2in}}}{\end{list}}

\usepackage{array, booktabs, caption}
\usepackage{makecell}
\usepackage{comment}
\newcommand*{\nl}{\newline}

\begin{document}

\name{Yifeng He \vspace*{.1in}}

\begin{resume}
\section{\sc Contact Information}
\vspace{.05in}
\textbf{Cell:} (530) 302-6806 |
\textbf{E-mail:} yfhe@ucdavis.edu |
\textbf{WWW:} \url{https://yfhe.net/}


\section{\sc Research Interests}
Program Analysis (Compilers, Types, and Fuzzing),
Code Understanding via Machine Learning \& NLP,
Programming Languages for Machine Learning \& AI,
Secure and Formal Verifiable Programs,
Computer Security.

\begin{comment}
\section{\sc EDUCATION}
\textbf{University of California, Davis}, Davis, California, USA  \\
\begin{itemize}
	\item Ph.D. in Computer Science \hfill {09/2023 -- present} \\
	\item B.S. in Computer Science \& Applied Mathematics \hfill {08/2019 -- 06/2023} \\
\end{itemize}

\section{\sc RESEARCH}
\textbf{Computer Security Lab at UC Davis} \hfill 10/2022 - present \\
\begin{itemize}
	\item Work with one PhD candidate, researchers from Tencent AI Lab, and Prof. Hao Chen on the frontend of programming language embedding.
	\item Generate fuzzing data (program-level IO pairs) for fine-tuning (Fuzz-tuning) LLM and achieved SOTA on downstream tasks.
	\item Generate fuzzing data (program-level IO pairs) for pre-training (Fuzz-pretrain) LLM.
	\item Proposed new tasks for evaluating testcase generation by LLM.
	\item Instrument open-source projects to generate function-level fuzzing IO-pairs via LLVM Pass.
\end{itemize}

\textbf{Path Academics} \hfill 02/2022 - 07/2022 \\
\begin{itemize}
	\item Conducted research on neural network and its application in AI under the supervision of Prof. Pavlos Protopapas from Harvard 
	\item Attended workshops on gradient descent algorithm, neural network optimizers,
		regularization of neural network, and other related concepts and architecture 
	\item Analyzed and compared models of deep learning algorithms application, 
	\item Made automatic differentiation to activation functions by hand, visualized receptive fields through max-pooling
\end{itemize}
\end{comment}

\section{\sc PUBLICATIONS}
Huang, J., Zhao, J., Rong, Y., Guo, Y., \textbf{He, Y.}, Chen, H.
\textit{Code Representation Pre-training with Complements from Program Executions},
International Conference on Learning Representations (ICLR), 2024. \textit{Under Review}.

Zhao, J., Rong, Y., Guo, Y., \textbf{He, Y.}, Chen, H. \textit{Understanding Programs by Exploiting (Fuzzing) Test Cases},
Findings of Association for Computational Linguistics (ACL), 2023.

\textbf{He, Y}, \textit{Big Data and Deep Learning Techniques Applied in Intelligent Recommender Systems}, 
IEEE 4th International Conference on Civil Aviation Safety and Information Technology (ICCASIT), 2022.

\section{\sc HONORS AND AWARDS} 
\textbf{Citation for Outstanding Performance}, Dept. Mathematics,  UC Davis, 2023 \\
\vspace{-0.5cm}

\textbf{Dean's Honor List}, College of L\&S, UC Davis, Fall 2019, Spring 2020, Spring 2021, Spring 2022
\vspace{0.5cm}

\section{\sc INTERNSHIP}
\textbf{ByteDance} \hfill {04/2021 -- 08/2021} \\
\textit{Software Engineering Intern}, {Income Platform Team}
\begin{itemize}
	\item Used microservice tech to connect parts of the author income settlement bushiness
	\item Transformed author-relation data architecture design from relational database (SQL) to graph database (Gremlin) to allow better efficiency for the business model
	\item Refactored the income calculation control process with visitor design pattern using Python 3
\end{itemize}

\textit{Software Engineering Intern}, {\textbf{Xigua Video} Author Experience Team}
\begin{itemize}
	\item Created a data cleaner script with ORM to maintain the size and readability of online data settlement table 
	\item Created the offline flow of Medium Video Encouragement Project for weekly data calculation
	\item Built the interface for frontend of web and mobile app to display the data visualization 
\end{itemize}

\begin{comment}
	
\section{\sc PROJECTS}
\textbf{CourseReco} \hfill {06/2022 -- 09/2022} \\
\begin{itemize}
	\item Designed the overall system architecture
	\item Led the programming for API server and recommender engine
	\item Negotiated with the third-party provider, SchedGo, for data service
	\item Provided technical leadership to teammates 
\end{itemize}

\textbf{Music Genre Classifier} \hfill {05/2022 -- 06/2022} \\
\begin{itemize}
	\item Processed music samples into spectrogram by Short-time Fourier transform
	\item Designed the appropriate model (CNN) to classify spectrograms into category
	\item Analyzed the resulting model and test outputs with saliency maps
\end{itemize}

\textbf{ImageOrientation} \hfill {03/2022 - 04/2022} \\
\begin{itemize}
	\item Pre-processed image data by rotating them with random generated angles, and assigned these angles as label
	\item Designed the appropriate CNN for regression task, tested and improved the model
	\item Applied Hyper-parameter tuning based on train, validation, and tested results 
\end{itemize}

\textbf{Dcash-server} \hfill {05/2021 - 07/2021} \\
\begin{itemize}
	\item Created a multi-threaded API server using C++ to allow users to create accounts to make deposit and transfer
	\item Used MySQL to store and maintain user data
	\item Made API calls to the Stripe API server to handle credit card information
\end{itemize}

\textbf{Genshine Impact Gacha Analyzer} \hfill {08/2021 - 09/2021} \\
\begin{itemize}
	\item Designed fetching process of gacha data from MiHoYo and categorized the process 
	\item Stored data into local database automatically, wrote into excel for data analysis by option
	\item Generated text or graph visualization report from data analyze results
\end{itemize}

\end{comment}

\section{\sc ACTIVITIES}
\textbf{HackerHub Club}, UC Davis \hfill {07/2020 - 06/2023} \\
\textit{Co-founder, President, Technical Officer}
\begin{itemize}
	\item Design and maintain a course recommendation system, CourseReco, for UC Davis students 
	\item Organize and lead the Code Jam Competition on data visualization, AI, augmented reality and virtual reality, and machine learning 
	\item Coach in introductory programming workshops, including Assembly, functional programming, recommender system, generative adversarial network, etc. 
\end{itemize}





\end{resume}
\end{document}




